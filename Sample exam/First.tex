\documentclass[10pt,a4paper]{article}
\usepackage[utf8]{inputenc}
\usepackage[english]{babel}
\usepackage{amsmath}
\usepackage{amsfonts}
\usepackage{amssymb}
\usepackage{amsthm}
\usepackage{enumitem}
\usepackage{listings}
\usepackage{pdfpages}
\usepackage{comment}
\usepackage{mathpartir}
\usepackage[top=2cm, bottom=3cm, left=1.5cm, right=1.5cm]{geometry}

\newcommand{\tu}{\textunderscore}

\lstset{
	tabsize=2,
	basicstyle=\small\ttfamily,
	columns=fixed,
	showstringspaces=false,
	breaklines=true,
	showtabs=false,
	showspaces=false,
	showstringspaces=false,
}

\newcommand{\courseCode}{Development - Databases}
%\newcommand{\git}{ \texttt{Dev81718First} }

\newcounter{ExerciseCount}
\setcounter{ExerciseCount}{1}

\newcommand{\functionEx}[3]{
  Implement a function
  
  \vspace{0.15cm}
  \texttt{let #1 #2 = ...}
   
   \vspace{0.15cm}
   #3
}
\newcommand{\emptyPlace}{\tu\tu\tu}
\newcommand{\exercise}[1]{\noindent \textbf{Exercise \theExerciseCount:}
  
  \vspace{0.15cm}
 #1 \addtocounter{ExerciseCount}{1}
}

\title{\courseCode}
\date {  }


\begin{document}
%\includepdf{Cover/first}
\maketitle

\exercise{
  \noindent
   Consider the following schema definition:
    
  \begin{table}[!h]
  \centering
    \begin{tabular}{|c|c|c|c|c|}
    \hline
    \multicolumn{5}{|c|}{\textbf{Author}}\\
    \hline
    AuthorId : int & Name : varchar(20) & LastName : varchar(20) & BirthYear : int & DeathYear : int\\
    \hline
    \end{tabular}
  \end{table}
  
   \begin{table}[!h]
       \centering
         \begin{tabular}{|c|c|c|c|c|}
         \hline
         \multicolumn{5}{|c|}{\textbf{Book}}\\
         \hline
         BookId : int & Title : varchar(20) & Pages : int & Year : int & Edition : smallint\\
         \hline
         \end{tabular}
     \end{table}
     
     \begin{table}[!h]
       \centering
         \begin{tabular}{|c|c|}
         \hline
         \multicolumn{2}{|c|}{\textbf{Author\tu Book}}\\
         \hline
         BookId : int & AuthorId : int\\
         \hline
         \end{tabular}
     \end{table}
     
     \noindent
     with the following constraints:
     
     \begin{table}[!h]
       \centering
       \begin{tabular}{|c|c|c|}
          \hline
          \textbf{table} & \textbf{attributes} & \textbf{constraint}\\
          \hline
          Author & AuthorId & primary key\\
          \hline
          Book & BookId & primary key\\
          \hline
          Author\tu Book & BookId,AuthorId & primary key\\
          \hline
          Author\tu Book & BookId & foreign key referencing Book\\
          \hline
          Author\tu Book & AuthorId & foreign key referencing Author\\
          \hline
       \end{tabular}
     \end{table}
 
     \noindent
     Implement a query that retrieves all the books written by J.R.R. Tolkien in their first edition with less than 1000 pages.
}
\end{document}